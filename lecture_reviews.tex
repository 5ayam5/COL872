\mathchardef\mhyphen="2D
\documentclass[11pt]{article}
\usepackage[english]{babel}
\usepackage{minted}
\usepackage{amsfonts}
\usepackage{amsmath}
\usepackage{amsthm}
\usepackage{amssymb}
\usepackage{graphicx}
\usepackage{subcaption}
\usepackage[hypcap=false]{caption}
\usepackage{booktabs}
\usepackage[left=25mm, top=25mm, bottom=25mm, right=25mm]{geometry}
\usepackage{soul}
\usepackage{algorithm}
\usepackage{algpseudocode}
\usepackage[most]{tcolorbox}
\usepackage[colorlinks=true,linkcolor=darkcyan,filecolor=darkcerulean,urlcolor=magenta]{hyperref}
\usepackage{braket}
\usepackage{quantikz}

\newcommand{\calA}{\mathcal{A}}
\newcommand{\calB}{\mathcal{B}}
\newcommand{\calC}{\mathcal{C}}
\newcommand{\calD}{\mathcal{D}}
\newcommand{\calE}{\mathcal{E}}
\newcommand{\calF}{\mathcal{F}}
\newcommand{\calG}{\mathcal{G}}
\newcommand{\calH}{\mathcal{H}}
\newcommand{\calI}{\mathcal{I}}
\newcommand{\calJ}{\mathcal{J}}
\newcommand{\calK}{\mathcal{K}}
\newcommand{\calL}{\mathcal{L}}
\newcommand{\calM}{\mathcal{M}}
\newcommand{\calN}{\mathcal{N}}
\newcommand{\calO}{\mathcal{O}}
\newcommand{\calP}{\mathcal{P}}
\newcommand{\calQ}{\mathcal{Q}}
\newcommand{\calR}{\mathcal{R}}
\newcommand{\calS}{\mathcal{S}}
\newcommand{\calT}{\mathcal{T}}
\newcommand{\calU}{\mathcal{U}}
\newcommand{\calV}{\mathcal{V}}
\newcommand{\calW}{\mathcal{W}}
\newcommand{\calX}{\mathcal{X}}
\newcommand{\calY}{\mathcal{Y}}
\newcommand{\calZ}{\mathcal{Z}}

\newcommand{\bfA}{\mathbf{A}}
\newcommand{\bfB}{\mathbf{B}}
\newcommand{\bfC}{\mathbf{C}}
\newcommand{\bfD}{\mathbf{D}}
\newcommand{\bfE}{\mathbf{E}}
\newcommand{\bfI}{\mathbf{I}}
\newcommand{\bfS}{\mathbf{S}}
\newcommand{\bfP}{\mathbf{P}}
\newcommand{\bfQ}{\mathbf{Q}}
\newcommand{\bfU}{\mathbf{U}}
\newcommand{\bfv}{\mathbf{v}}
\newcommand{\bfu}{\mathbf{u}}
\newcommand{\bfdelta}{\mathbf{\Delta}}
\newcommand{\bfpi}{\mathbf{\Pi}}


\newcommand{\N}{\mathbb{N}}
\newcommand{\z}{\mathbb{Z}}
\newcommand{\I}{\mathbb{I}}
\newcommand{\C}{\mathbb{C}}

\newcommand{\keygen}{\mathsf{KeyGen}}
\newcommand{\enc}{\mathsf{Enc}}
\newcommand{\dec}{\mathsf{Dec}}
\newcommand{\negl}{\mathsf{negl}}
\newcommand{\commit}{\mathsf{Commit}}
\newcommand{\ccommit}{\mathsf{C\text{-}Commit}}

\newcommand{\setup}{\mathsf{Setup}}
\newcommand{\lsetup}{\mathsf{Setup\text{-}Lossy}}

\newcommand{\samplemat}{\mathsf{Sample\text{-}Dirac\text{-}Matrix}}
\newcommand{\eval}{\mathsf{Eval}}
\newcommand{\obf}{\mathsf{Obf}}

\newcommand{\phybb}[1]{p_{\mathrm{hyb}, #1}}
\newcommand{\lin}{\ell_{\mathrm{in}}}
\newcommand{\lout}{\ell_{\mathrm{out}}}
\newcommand{\bit}{\{0,1\}}
\newcommand{\sd}{\mathsf{SD}}
\newcommand{\lwe}{\mathsf{LWE}_{n,m,q,\chi}}
\newcommand{\sslwe}[4]{\mathsf{ss\text{-}LWE}_{#1,#2,#3,#4}}
\newcommand{\sslwec}{\sslwe{n}{m}{q}{\chi}}

\newcommand{\unif}[1]{\mathsf{Unif}_{\left[-#1, #1\right]}}
\newcommand{\func}[2]{\mathsf{Func}[#1, #2]}
\newcommand{\perm}[2]{\mathsf{Perm}[#1, #2]}
\newcommand{\ct}{\mathsf{ct}}
\newcommand{\Finverse}{F^{-1}}

\newcommand{\nqss}{\mathsf{No\mhyphen Query \mhyphen Semantic \mhyphen Security}}

\newcommand{\Tr}{\mathrm{Tr}}
\newcommand{\trace}[1]{\Tr\left( #1 \right)}

\newcommand{\prob}[1]{\Pr\left[ #1 \right]}

\newcommand{\ketbraa}[2]{\ket{#1}\bra{#2}}
\newcommand{\ketbra}[1]{\ketbraa{#1}{#1}}

\newcommand{\lddh}{\mathcal{L}_{DDH}}
\newcommand{\lnddh}{\mathcal{L}_{nDDH}}
\newcommand{\lddhkt}{\mathcal{L}_{DDH,k,t}}

\newlength{\protowidth}
\newcommand{\pprotocol}[4]{
{\begin{center}
\setlength{\protowidth}{\textwidth}
\addtolength{\protowidth}{-3\intextsep}

\fbox{
        \small
        \hbox{\quad
        \begin{minipage}{\protowidth}
    \begin{center}
    {\bf #1}
    \end{center}
        #4
        \end{minipage}
        \quad}
        }
        \captionof{figure}{\label{#3} #2}
\end{center}
} }

\newcommand{\defbox}[1]{
{\begin{center}
\setlength{\protowidth}{\textwidth}
\addtolength{\protowidth}{-3\intextsep}

\fcolorbox{darkcerulean}{cottoncandy}{
        \small
        \hbox{\quad
        \begin{minipage}{\protowidth}
    
        #1
        \end{minipage}
        \quad}
        }
\end{center}
        } }

\newcommand{\protocol}[4]{
\pprotocol{#1}{#2}{#3}{#4} }

\newtheorem{theorem}{Theorem}[section]
\newtheorem{claim}[theorem]{Claim}
\newtheorem{fact}[theorem]{Fact}
\newtheorem{definition}[theorem]{Definition}
\newtheorem*{question}{Question}

\newtcolorbox{solution}[2][]{every float=\centering,breakable,enhanced,adjusted title={#2},colback=codegray,colframe=codegray!50!black}

\linespread{1.0}

\definecolor{codegray}{rgb}{0.98,0.97,0.93}
\definecolor{cottoncandy}{rgb}{1.0, 0.84, 0.95}
\definecolor{darkcerulean}{rgb}{0.03, 0.27, 0.49}
\definecolor{darkcyan}{rgb}{0.0, 0.50, 0.45}

\title{COL872\\Lecture Reviews}
\author{Sayam Sethi (2019CS10399)}
\date{May 2023}

\begin{document}

\maketitle

\tableofcontents

\pagenumbering{arabic}

\section{Lecture $13$}
\begin{itemize}
    \item When introducing quantum oracle (first section on page $70$), it would be helpful if an intuition can be given as to why the oracle is defined this way (and why it is defined as a unitary; why it is not defined as a unitary followed by some measurement).
    \item Although the equivalence between standard oracle and phase oracle was given in the assignment (for a single bit output and assuming that the second register is $\ket{-}$), it would be great if the generic version of the same can also be stated after introducing the phase-kickback trick on page $71$ (PS: when reading Zhandry's paper, even though the transformation is trivial, it took us a little bit of time to think of and interpret the proof/transformation - partly because the extension from the single bit to multi-bit version involves a dot product which is not directly apparent in the case of a single bit output).
    \item I feel that it would be helpful to include a short discussion on the difference (or similarity) between the computational complexity of a single query in both the classical and quantum settings (and whether comparing the number of queries is enough to determine more efficient algorithms).
    \item ``If $f$ is a constant function, then this state is \textcolor{red}{either $\ket{+}^{\otimes n}$ or $-\ket{+}^{\otimes n}$}" (page $72$)
\end{itemize}

\section{Lecture $14$}
\begin{itemize}
    \item Isn't necessary, but it might be a good idea to state that no classical algorithm can do better than $\Theta(n)$ for the `Bernstein-Vazirani Problem' (or just state that any classical algorithm requires $\Omega(n)$ time).
    \item It would be better if the variable $y$ is changed to a different variable in the definition of Simon's problem since in the later discussion $y$ is used for output.
    \item The proposed quantum algorithm for Simon's problem won't work when $\mathbf{s} = 0^n$ although this is not explicitly mentioned.
    \item I probably missed the intuition for this, however, it took me a lot of lectures to be comfortable with easily being able to argue when quantum states are orthogonal or not. For both `Deutsch-Josza Problem' (page $72$: \textcolor{blue}{If $f$ is a balanced function, then $\ldots$ is orthogonal $\ldots$}) and `Simon's Problem' (page $74$: \textcolor{blue}{Note that the only terms $\ldots$ are orthogonal $\ldots$}), I wasn't able to fully accept the orthogonality claimed in the notes during those lectures. It might be helpful if a short sketch of the argument is provided.
\end{itemize}

\section{Lecture $16$}
\begin{itemize}
    \item (a slight formatting improvement) The `Basic Properties of Density Matrices' section has a lot of notation and hence it would be visually better if the three properties are highlighted or made bold so that they easily stand out.
    \item I feel that a small discussion on how to distinguish arbitrary (or special) density matrices would be helpful and would provide a helpful background in general as well as for the future lectures when we use alternating projectors. 
    \item \textcolor{green}{``Note to reviewer: is the above explanation (together with the exercise) sufficient?"}: The off-diagonal argument makes sense to me (under the assumption that the diagonal argument is true), however, the argument for diagonal entries isn't completely clear. Particularly, I am not sure why does the result follow from the reasoning given in brackets about `existence of an output'. The exercise given makes the argument for off-diagonal entries clear.
\end{itemize}

\end{document}
